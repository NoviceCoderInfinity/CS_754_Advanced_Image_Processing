\documentclass{article}
\usepackage{helvet}


\usepackage{cite}
\usepackage{amsmath,amssymb,amsfonts}
\usepackage{algorithmic}
\usepackage{graphicx}
\usepackage{textcomp}
\usepackage{xcolor}
\usepackage{hyperref}
\usepackage{placeins}
\usepackage{graphicx}
\usepackage{subcaption}



\def\BibTeX{{\rm B\kern-.05em{\sc i\kern-.025em b}\kern-.08em
    T\kern-.1667em\lower.7ex\hbox{E}\kern-.125emX}}


\title{Assignment 1: CS 754, Spring 2024-25}
\author{
\IEEEauthorblockN{
    \begin{tabular}{cccc}
        \begin{minipage}[t]{0.23\textwidth}
            \centering
            Amitesh Shekhar\\
            IIT Bombay\\
            22b0014@iitb.ac.in
        \end{minipage} & 
        \begin{minipage}[t]{0.23\textwidth}
            \centering
            Anupam Rawat\\
            IIT Bombay\\
            22b3982@iitb.ac.in
        \end{minipage} & 
        \begin{minipage}[t]{0.23\textwidth}
            \centering
            Toshan Achintya Golla\\
            IIT Bombay\\
            22b2234@iitb.ac.in
        \end{minipage} \\
        \\ 
    \end{tabular}
}
}

\date{February 01, 2025}


\usepackage{amsmath}
\usepackage{amssymb}
\usepackage{hyperref}
\usepackage{ulem,graphicx}
\usepackage[margin=0.5in]{geometry}

\begin{document}
\maketitle

\\
\\

\textbf{Declaration:} The work submitted is our own, and
we have adhered to the principles of academic honesty while completing and submitting this work. We have not
referred to any unauthorized sources, and we have not used generative AI tools for the work submitted here.

\begin{enumerate}
    \item 
         Consider a $m \times n$ sensing matrix $\boldsymbol{A}$ ($m < n$) with order-$s$ restricted isometry constant (RIC) of $\delta_s$. Let $\mathcal{S}$ be a subset of up to $s$ elements from $\{1,2,...,n\}$. Let $\boldsymbol{A}_{\mathcal{S}}$ be a $m  \times |S|$ sub-matrix of $\boldsymbol{A}$ with columns corresponding to indices in $\mathcal{S}$. Let $\lambda_{max}$ be the maximum of the maximal eigenvalue of any matrix $\boldsymbol{A}^T_{\mathcal{S}} \boldsymbol{A}_{\mathcal{S}}$ (i.e. the maximum is taken across all possible subsets of size up to $s$). Let $\lambda_{min}$ be the minimum of the minimal eigenvalue of any matrix $\boldsymbol{A}^T_{\mathcal{S}} \boldsymbol{A}_{\mathcal{S}}$ (i.e. the minimum is taken across all possible subsets of size up to $s$). Then prove that $\delta_s = \textrm{max}(1-\lambda_{min},\lambda_{max}-1)$. \textsf{[15 points]}
        
        \\
            \makebox[0pt][l]{\hspace{-7pt}\textit{Soln:}} % Aligns "Answer:" to the left
        \\
        $A_{MxN}$ is a sensing matrix which obeys RIP of order-s with RIC of $\delta_s $.\\
        $\implies$
        \[
            (1-\delta_s )\ ||\theta||^{2} \leq ||A \theta||^{2} \leq (1+\delta_s)\ ||\theta||^{2}
        \]
        where $\theta$ is a s-sparse vector of dimension $n\times1$.\\
        Now, we can extend this to include any sub-matrix $A_{S}$ of A of size $m \times |S|$ where S is a subset of $\{1,2,...,n\}$ of cardinality at-most s.\\
        \[
            (1-\delta_s )\ ||\theta||^{2} \leq ||A_{S} \theta||^{2} \leq (1+\delta_s)\ ||\theta||^{2}
        \]
        $\implies$
        
        \begin{equation}
            (1-\delta_s )\ \leq \frac{||A_{S} \theta||^{2}}{||\theta||^{2}} \leq (1+\delta_s)
        \end{equation}
        We have seen in class that for a symmetric matrix $B^{T}B$, its maximum and minimum eigen-value $\lambda_{max}$ is given by:
        \[
            \lambda_{max} = \underset{x}{\max}  \frac{||Bx||^2}{||x||^2}
        \]
        \[
            \lambda_{min} = \underset{x}{\min}  \frac{||Bx||^2}{||x||^2}
        \]
        where x is a column vector of compatible dimensions.\\

        \vspace{2em}
        Now, looking at the left-side inequality of equation (1), we obtain
        \[ 
            (1-\delta_s )\ \leq \frac{||A_{S} \theta||^{2}}{||\theta||^{2}}
        \]
        For this inequality to hold for all subsets S, $(1-\delta_s)$ must be smaller than $\underset{S, \theta}{min} \frac{||A_{S} \theta||^{2}}{||\theta||^{2}}$. \\
        But this quantity is equal to $\lambda_{min}$ (according to the definition of minimal eigen-value).\\

        \begin{align}
        \therefore (1-\delta_s) \leq \lambda_{min}\\
        \implies \delta_s \geq 1-\lambda_{min}
        \end{align}
        
        Similarly, extending this argument to the right-side inequality of equation (1), we get:
        \begin{align}
        (1+\delta_s) \geq \lambda_{max}\\
        \implies \delta_s \geq \lambda_{max} -1
        \end{align}

        Finally, From equations (3) and (5), we get:
        \begin{align}
            \delta_s \geq max(1-\lambda_{min}, \lambda_{max}-1)
        \end{align}
        Since, by definition, the RIC $\delta_s$ is taken to be the smallest possible constant satisfying the RIP property, we can replace the $\geq$ sign by an $=$\\
        Therefore, 
        \begin{equation}
                 \delta_s = max(1-\lambda_{min}, \lambda_{max}-1)
        \end{equation}
        (proved)
        
        
\end{enumerate}
\end{document}