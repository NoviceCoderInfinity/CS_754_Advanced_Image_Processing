\documentclass{article}
\usepackage{helvet}


\usepackage{cite}
\usepackage{amsmath,amssymb,amsfonts}
\usepackage{algorithmic}
\usepackage{graphicx}
\usepackage{textcomp}
\usepackage{xcolor}
\usepackage{hyperref}
\usepackage{placeins}
\usepackage{graphicx}
\usepackage{subcaption}



\def\BibTeX{{\rm B\kern-.05em{\sc i\kern-.025em b}\kern-.08em
    T\kern-.1667em\lower.7ex\hbox{E}\kern-.125emX}}


\title{Assignment 1: CS 754, Spring 2024-25}
\author{
\IEEEauthorblockN{
    \begin{tabular}{cccc}
        \begin{minipage}[t]{0.23\textwidth}
            \centering
            Amitesh Shekhar\\
            IIT Bombay\\
            22b0014@iitb.ac.in
        \end{minipage} & 
        \begin{minipage}[t]{0.23\textwidth}
            \centering
            Anupam Rawat\\
            IIT Bombay\\
            22b3982@iitb.ac.in
        \end{minipage} & 
        \begin{minipage}[t]{0.23\textwidth}
            \centering
            Toshan Achintya Golla\\
            IIT Bombay\\
            22b2234@iitb.ac.in
        \end{minipage} \\
        \\ 
    \end{tabular}
}
}

\date{February 01, 2025}


\usepackage{amsmath}
\usepackage{amssymb}
\usepackage{hyperref}
\usepackage{ulem,graphicx}
\usepackage[margin=0.5in]{geometry}

\begin{document}
\maketitle

\\
\\

\textbf{Declaration:} The work submitted is our own, and
we have adhered to the principles of academic honesty while completing and submitting this work. We have not
referred to any unauthorized sources, and we have not used generative AI tools for the work submitted here.

\begin{enumerate}
    \item This homework problem is inspired from one of the questions asked to me in class. Consider a signal $g$ with $n$ elements where $g$ is the Dirac comb consisting of spikes separated by $\sqrt{n}$ in time. Let $F$ be the support set of $g$ in the Fourier domain (i.e. the set of frequencies at which its Fourier transform is non-zero), and let $\Omega$ be the set of frequencies at which the Fourier transform of $g$ is measured. Let us assume that $\Omega$ is chosen uniformly at random. We want to derive lower bounds on the size of $\Omega$ in order to be able to reconstruct $g$ exactly from these measurements with a probability of at least $1-n^{-M}$ where $M > 0$. To this end, answer the following questions. Do not merely quote theorems or results, but answer this from first principles: \textsf{[5+5+5=15 points]}

    \begin{enumerate}
            \item
                If the intersection of $\Omega$ with $F$ is a null set, then we definitely have no chance of recovering $g$. What is the probability of this happening in terms of $|\Omega|, n, |F|$? Here $|F|$ stands for the cardinality of $F$.
            \\
    
        \makebox[0pt][l]{\hspace{-7pt}\textit{Soln:}} % Aligns "Answer:" to the left
    \\
    We have total n frequencies, out of which g is non-zero at ${|F|}$ points. We want to select $|\Omega|$ number of points such that $\Omega$ and $F$ are disjoint sets. The probability can therefore be written as:
    \[
    \frac{\binom{n-|F|}{|\Omega|}}{\binom{n}{|\Omega|}}
    \]
            \item 
                Argue that this probability is \textbf{upper bounded} by $(1-|\Omega|/n)^{|F|}$.
            \\
    
        \makebox[0pt][l]{\hspace{-7pt}\textit{Soln:}} % Aligns "Answer:" to the left
    \\
    We can expand the probability obatined in part (a) as follows:
    \[
    \frac{(n-|F|)! (n-|\Omega|)!}{(n-|F|-|\Omega|)! n!}
    \]
    This gives:
    \[
    \frac{(n-|\Omega|)(n-|\Omega|-1).........(n-|F|-|\Omega|+1)}{n(n-1)..........(n-|F|+1)}
    \]
    which further simplifies to:
    \[
    (1-\frac{|\Omega|}{n})(1-\frac{|\Omega|}{n-1})............(1-\frac{|\Omega|}{n-|F|+1})
    \]
    which is \textbf{upper bounded} by \[
    (1-\frac{|\Omega|}{n})^{|F|}
    \]
    
            \item 
                Hence derive a lower bound on $|\Omega|$. Use the assumption that $|\Omega| \ll n$ so that $\log(1-|\Omega|/n) \approx -|\Omega|/n$.
        \\
    
        \makebox[0pt][l]{\hspace{-7pt}\textit{Soln:}} % Aligns "Answer:" to the left
    \\
    (c) To ensure exact reconstruction of \(g\) with probability at least \(1 - n^{-M}\), we require:

\[
P(\Omega \cap F = \emptyset) \leq n^{-M}.
\]

Using the \textbf{upper bound} from part (b):

\[
\left(1 - \frac{|\Omega|}{n}\right)^{|F|} \leq n^{-M}.
\]

Taking the natural logarithm on both sides and using the approximation \(\log(1 - x) \approx -x\) for \(x \ll 1\) (since \(|\Omega| \ll n\)):

\[
|F| \cdot \log\left(1 - \frac{|\Omega|}{n}\right) \leq -M \log n.
\]

Substituting \(\log(1 - \frac{|\Omega|}{n}) \approx -\frac{|\Omega|}{n}\):

\[
|F| \cdot \left(-\frac{|\Omega|}{n}\right) \leq -M \log n.
\]

Simplifying:

\[
\frac{|\Omega||F|}{n} \geq M \log n.
\]

Thus, the lower bound on \(|\Omega|\) is:

\[
|\Omega| \geq \frac{M n \log n}{|F|}.
\]

    
        \end{enumerate}
\end{enumerate}
\end{document}