\documentclass{article}
\usepackage{helvet}


\usepackage{cite}
\usepackage{amsmath,amssymb,amsfonts}
\usepackage{algorithmic}
\usepackage{graphicx}
\usepackage{textcomp}
\usepackage{xcolor}
\usepackage{hyperref}
\usepackage{placeins}
\usepackage{graphicx}
\usepackage{subcaption}



\def\BibTeX{{\rm B\kern-.05em{\sc i\kern-.025em b}\kern-.08em
    T\kern-.1667em\lower.7ex\hbox{E}\kern-.125emX}}


\title{Assignment 1: CS 754, Spring 2024-25}
\author{
\IEEEauthorblockN{
    \begin{tabular}{cccc}
        \begin{minipage}[t]{0.23\textwidth}
            \centering
            Amitesh Shekhar\\
            IIT Bombay\\
            22b0014@iitb.ac.in
        \end{minipage} & 
        \begin{minipage}[t]{0.23\textwidth}
            \centering
            Anupam Rawat\\
            IIT Bombay\\
            22b3982@iitb.ac.in
        \end{minipage} & 
        \begin{minipage}[t]{0.23\textwidth}
            \centering
            Toshan Achintya Golla\\
            IIT Bombay\\
            22b2234@iitb.ac.in
        \end{minipage} \\
        \\ 
    \end{tabular}
}
}

\date{February 01, 2025}


\usepackage{amsmath}
\usepackage{amssymb}
\usepackage{hyperref}
\usepackage{ulem,graphicx}
\usepackage[margin=0.5in]{geometry}

\begin{document}
\maketitle

\\
\\

\textbf{Declaration:} The work submitted is our own, and
we have adhered to the principles of academic honesty while completing and submitting this work. We have not
referred to any unauthorized sources, and we have not used generative AI tools for the work submitted here.

\begin{enumerate}
    \item 
        For a unique solution to the P1 problem, we require that $\delta_{2s} < 0.41$ as given in class. What is the corresponding upper bound for $\delta_{2s}$ in order for the P0 problem to give a unique solution? (Hint: Look at the proof of the uniqueness of the solutions to the P0 problem, and see the definition of RIC) \textsf{[15 points]}
        \\
            \makebox[0pt][l]{\hspace{-7pt}\textit{Soln:}} % Aligns "Answer:" to the left
        \\
        The P0 problem requires us to minimize $\| \theta\|_0$ for $y = A\theta$. \\
        To find the upper bound of $\delta_{2s}$ in order for the P0 problem to give a unique solution, we begin by assuming that there exist 2 solutions $\theta_1$ and $\theta_2$; such that $\theta_1 \neq \theta_2$. Thus, 
        \[
            \| \theta_1 \|_0 = \| \theta_2 \|_0 = S
        \]
        \[
            y = A\theta_1 = A\theta_2
        \]
        \[
            A(\theta_1 - \theta_2) = 0
        \]
        Thus, either the below is true
        \[
            \theta_1 = \theta_2
        \]
        or the below is true
        \[
            \theta_1 - \theta_2 \in nullspace(A)
        \]
        As per our beginning statement, $\theta_1 \neq \theta_2$, thus
        \[
            \| \theta_1 - \theta_2 \|_0 \leq 2S
        \]
        Now,
        \[
            A(\theta_1 - \theta_2) \neq 0
        \]
        as A has Restrictive Isometric Property of 2S-order.
        \[
            (1 - \delta_S) \| \theta_{2S} \|^2 \leq \| A\theta_{2S}\|^2 \leq (1 + \delta_{2S})\|\theta_{2S}\|^2
        \]
        Replacing $\theta_{2S}$ with $\theta_1 - \theta_2$,
        \[
            (1 - \delta_S) \| \theta_1 - \theta_2 \|^2 \leq \| A (\theta_1 - \theta_2)\|^2 \leq (1 + \delta_{2S})\|\theta_1 - \theta_2\|^2
        \]
        For, this to not to be equal to zero,
        \[
            (1 - \delta_S) \| \theta_1 - \theta_2 \|^2 > 0
        \]
        Since, our definition $\theta_1 \neq \theta_2$
        \[
            (1 - \delta_{2S}) > 0
        \]
        \[
            \delta_{2S} < 1
        \]
        Thus, the upper bound for $\delta_{2s}$ in order for the P0 problem to give a unique solution
        
\end{enumerate}
\end{document}