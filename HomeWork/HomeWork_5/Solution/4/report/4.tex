\documentclass{article}
\usepackage{helvet}


\usepackage{cite}
\usepackage{amsmath,amssymb,amsfonts}
\usepackage{algorithmic}
\usepackage{graphicx}
\usepackage{textcomp}
\usepackage{xcolor}
\usepackage{hyperref}
\usepackage{placeins}
\usepackage{graphicx}
\usepackage{subcaption}



\def\BibTeX{{\rm B\kern-.05em{\sc i\kern-.025em b}\kern-.08em
    T\kern-.1667em\lower.7ex\hbox{E}\kern-.125emX}}


\title{Question 4, Assignment 5: CS 754, Spring 2024-25}
\author{
\IEEEauthorblockN{
    \begin{tabular}{cccc}
        \begin{minipage}[t]{0.23\textwidth}
            \centering
            Amitesh Shekhar\\
            IIT Bombay\\
            22b0014@iitb.ac.in
        \end{minipage} & 
        \begin{minipage}[t]{0.23\textwidth}
            \centering
            Anupam Rawat\\
            IIT Bombay\\
            22b3982@iitb.ac.in
        \end{minipage} & 
        \begin{minipage}[t]{0.23\textwidth}
            \centering
            Toshan Achintya Golla\\
            IIT Bombay\\
            22b2234@iitb.ac.in
        \end{minipage} \\
        \\ 
    \end{tabular}
}
}

\date{April 15, 2025}


\usepackage{amsmath}
\usepackage{amssymb}
\usepackage{hyperref}
\usepackage{ulem,graphicx}
\usepackage[margin=0.5in]{geometry}

\begin{document}
\maketitle

\\


\begin{enumerate}
\item Consider that you learned a dictionary $\boldsymbol{D}$ to sparsely represent a certain class $\mathcal{S}$ of images - say handwritten alphabet or digit images. How will you convert $\boldsymbol{D}$ to another dictionary which will sparsely represent the following classes of images? Note that you are not allowed to learn the dictionary all over again, as it is time-consuming. 
\begin{enumerate}
\item Class $\mathcal{S}_1$ which consists of images obtained by applying a known affine transform $\boldsymbol{A_1}$ to a subset of the images in class $\mathcal{S}$, and by applying another known affine transform $\boldsymbol{A_2}$ to the other subset. Assume that the images in $\mathcal{S}$ consisted of a foreground against a constant 0-valued background, and that the affine transformations $\boldsymbol{A_1}, \boldsymbol{A_2}$ do not cause the foreground to go outside the image canvas. 
\item Class $\mathcal{S}_2$ which consists of images obtained by applying an intensity transformation $I^i_{new}(x,y) = \alpha (I^i_{old}(x,y))^2 + \beta (I^i_{old}(x,y)) + \gamma$ to the images in $\mathcal{S}$, where $\alpha,\beta,\gamma$ are known.  
\item Class $\mathcal{S}_4$ which consists of images obtained by downsampling the images in $\mathcal{S}$ by a factor of $k$ in both X and Y directions. 
\item Class $\mathcal{S}_5$ which consists of images obtained by applying a blur kernel which is known to be a linear combination of blur kernels belonging to a known set $\mathcal{B}$, to the images in $\mathcal{S}$. 
\item Class $\mathcal{S}_6$ which consists of 1D signals obtained by applying a Radon transform in a known angle $\theta$ to the images in $\mathcal{S}$. 
\textsf{[4 x 5 = 20 points]}
\end{enumerate}
\end{enumerate}
\noindent (a) Let $\boldsymbol{A}_1$ and $\boldsymbol{A}_2$ be the known affine transformation operators. Since affine transforms are linear, they can be applied to each atom in the dictionary. Let $D_{A_1}$ and $D_{A_2}$ be the transformed dictionaries obtained by applying $\boldsymbol{A}_1$ and $\boldsymbol{A}_2$ respectively to every reshaped atom of $\boldsymbol{D}$. Then,
    \[
    D' = [D_{A_1} \; | \; D_{A_2}]
    \]
    This new dictionary $D'$ will be able to sparsely represent both subsets of images, as follows:
    \[
    f' = D' \begin{bmatrix}
        \theta_f \\ 0
    \end{bmatrix} \quad \text{if } f' \text{ came from } A_1(f),
    \quad \text{and} \quad
    f' = D' \begin{bmatrix}
        0 \\ \theta_f
    \end{bmatrix} \quad \text{if } f' \text{ came from } A_2(f)
    \]
    where $f = D \theta_f$ and $\theta_f$ is $k$-sparse.\\[1em]

\noindent (b) For an image $f = D\theta$, consider the intensity transformation:
    \[
    f' = \alpha (f \odot f) + \beta f + \gamma \mathbf{1}
    \]
    where $\odot$ denotes point-wise multiplication and $\mathbf{1}$ is a vector of all ones. Expanding:
    \[
    f' = \alpha (D\theta \odot D\theta) + \beta D\theta + \gamma \mathbf{1}
    \]
    Let $\widetilde{D}$ be a matrix formed by element-wise products of all pairs of dictionary atoms:
    \[
    \widetilde{D}_{(i,j)} = D_i \odot D_j \quad \forall i,j
    \]
    and $\widetilde{\theta} = \text{vec}(\theta\theta^T)$, then:
    \[
    f' = [\mathbf{1} \; | \; D \; | \; \widetilde{D}]
    \begin{bmatrix}
        \gamma \\ \beta \theta \\ \alpha \widetilde{\theta}
    \end{bmatrix}
    \]
    So, the new dictionary is:
    \[
    D' = [\mathbf{1} \; | \; D \; | \; \widetilde{D}]
    \]
    which contains the constant vector, the original dictionary atoms, and all pairwise element-wise products of atoms.\\[1em]

\noindent (c) Downsampling by a factor $k$ is a linear operation that reduces spatial resolution. Let $S_k$ be the downsampling operator that maps an image of size $(h, w)$ to $(h/k, w/k)$. Then, applying this to each atom in $D$ (after reshaping) gives:
    \[
    D'_i = \text{vec}(S_k(\text{reshape}(D_i))) \quad \forall i
    \]
    So the new dictionary is:
    \[
    D' = [D'_1 \; D'_2 \; \dots \; D'_K]
    \]
    where each $D'_i$ is a downsampled version of $D_i$.\\[1em]

\noindent (d) Suppose the blur kernel $b$ is given by $b = \sum_{j=1}^{n} \beta_j b_j$ for known $\{b_j\} \in \mathcal{B}$. Since convolution is linear:
    \[
    f' = b * \text{reshape}(f) = \left( \sum_j \beta_j b_j \right) * \left( \sum_i D_i \theta_i \right)
    = \sum_{i,j} \beta_j (b_j * \text{reshape}(D_i)) \theta_i
    \]
    Define $D^{(j)}$ as the dictionary obtained by convolving each atom $D_i$ (reshaped) with $b_j$:
    \[
    D^{(j)}_i = \text{vec}(b_j * \text{reshape}(D_i)) \quad \Rightarrow \quad D' = \sum_j \beta_j D^{(j)}
    \]
    So $D'$ is the linear combination of the blurred versions of the original dictionary atoms.\\[1em]

\noindent (e) Let $R_\theta$ denote the Radon transform at angle $\theta$. Since it's a linear operation:
    \[
    f' = R_\theta(\text{reshape}(f)) = R_\theta \left( \sum_i \text{reshape}(D_i) \theta_i \right)
    = \sum_i R_\theta(\text{reshape}(D_i)) \theta_i
    \]
    Let $D'_i = R_\theta(\text{reshape}(D_i))$, vectorised. Then:
    \[
    D' = [D'_1 \; D'_2 \; \dots \; D'_K]
    \]
    is the new dictionary, where each atom is transformed by the Radon operator.

\end{document}