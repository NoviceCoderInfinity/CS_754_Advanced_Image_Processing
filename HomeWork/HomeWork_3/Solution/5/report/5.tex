\documentclass{article}
\usepackage{helvet}


\usepackage{cite}
\usepackage{amsmath,amssymb,amsfonts}
\usepackage{algorithmic}
\usepackage{graphicx}
\usepackage{textcomp}
\usepackage{xcolor}
\usepackage{hyperref}
\usepackage{placeins}
\usepackage{graphicx}
\usepackage{subcaption}



\def\BibTeX{{\rm B\kern-.05em{\sc i\kern-.025em b}\kern-.08em
    T\kern-.1667em\lower.7ex\hbox{E}\kern-.125emX}}


\title{Question 5, Assignment 3: CS 754, Spring 2024-25}
\author{
\IEEEauthorblockN{
    \begin{tabular}{cccc}
        \begin{minipage}[t]{0.23\textwidth}
            \centering
            Amitesh Shekhar\\
            IIT Bombay\\
            22b0014@iitb.ac.in
        \end{minipage} & 
        \begin{minipage}[t]{0.23\textwidth}
            \centering
            Anupam Rawat\\
            IIT Bombay\\
            22b3982@iitb.ac.in
        \end{minipage} & 
        \begin{minipage}[t]{0.23\textwidth}
            \centering
            Toshan Achintya Golla\\
            IIT Bombay\\
            22b2234@iitb.ac.in
        \end{minipage} \\
        \\ 
    \end{tabular}
}
}

\date{March 21, 2025}


\usepackage{amsmath}
\usepackage{amssymb}
\usepackage{hyperref}
\usepackage{ulem,graphicx}
\usepackage[margin=0.5in]{geometry}

\begin{document}
\maketitle

\textbf{Declaration:} The work submitted is our own, and
we have adhered to the principles of academic honesty while completing and submitting this work. We have not
referred to any unauthorized sources, and we have not used generative AI tools for the work submitted here.

\begin{enumerate}
\item Consider two observed particle images $Q_1$ and $Q_2$ corresponding to a 3D density map, each in different 3D orientations and 2D shifts. Let $Q_1$ be obtained by translating a zero-shift particle image $P_1$ by $(\delta_{x1},\delta_{y1})$.  Let $Q_2$ be obtained by translating a zero-shift particle image $P_2$ by $(\delta_{x2},\delta_{y2})$. Note that $Q_1, Q_2$ are practically observed, whereas $P_1, P_2$ are not observed. Let the common line for the particle images $P_1, P_2$ pass through the origins of their respective coordinate systems at angles $\theta_1$ and $\theta_2$ with respect to their respective X axes. Derive a relationship between $\delta_{x1},\delta_{y1}, \theta_1, \delta_{x2},\delta_{y2}, \theta_2$ and some other observable property of the projection images. Explain how you will determine $\delta_{x1}, \delta_{y1}, \delta_{x2}, \delta_{y2}$ using this equation. Explain how you will extend this relationship to determine the shifts $\{(\delta_{xi},\delta_{yi})\}_{i=1}^N$ of the $N$ different projection images, and mention the number of knowns and unknowns. \textsf{[5+2+8+3=18 points]}
\\
    \makebox[0pt][l]{\hspace{-7pt}\textit{Soln:}} % Aligns "Answer:" to the left
\\
We are given two observed particle images, $Q_1$ and $Q_2$, which correspond to different 2D projections of a common 3D density map. These images are obtained from their respective zero-shift versions, $P_1$ and $P_2$, but are shifted by unknown translations:
\begin{itemize}
    \item $Q_1$ is obtained by shifting $P_1$ by $(\delta_{x1}, \delta_{y1})$.
    \item $Q_2$ is obtained by shifting $P_2$ by $(\delta_{x2}, \delta_{y2})$.
\end{itemize}

Note that each 2D projection of the 3D structure corresponds to a slice of the 3D Fourier transform of the object, as described by the Fourier Slice Theorem. Since these slices pass through the origin in Fourier space, any two such slices must intersect along a common 1D line.
Since both $P_1$ and $P_2$ are 2D projections of the same 3D object, they share a common 1D projection line in Fourier space. Similarly, $Q_1$ and $Q_2$ share a common 1D projection line in Fourier space.


Let $\theta_1$ and $\theta_2$ be the angles of the common line in the coordinate systems of $P_1$ and $P_2$. Thereby we can say that the common line equation states that for all frequency magnitudes $k$:
\begin{equation}
    F_1(k \cos\theta_1, k \sin\theta_1) = F_2(k \cos\theta_2, k \sin\theta_2)
\end{equation}
where $F_1$ and $F_2$ are the Fourier transforms of the zero-shift images $P_1$ and $P_2$.

Now, we know that when an image is shifted in real space by $(\delta_x, \delta_y)$, then the Fourier transform P of the given image and Fourier transform Q of the shifted image are related by a phase shift as follows:
\begin{equation}
    F_Q(k_x, k_y) = F_P(k_x, k_y) e^{-i( k_x \delta_x + k_y \delta_y )}
\end{equation}
Thus, the observed images $Q_1$ and $Q_2$ have Fourier transforms:
\begin{equation}
    F_{Q_1}(k_x, k_y) = F_{P_1}(k_x, k_y) e^{-i (k_x \delta_{x1} + k_y \delta_{y1})}
\end{equation}
\begin{equation}
    F_{Q_2}(k_x, k_y) = F_{P_2}(k_x, k_y) e^{-i (k_x \delta_{x2} + k_y \delta_{y2})}
\end{equation}
Using equations (1), (3) and (4), we get:
\begin{equation}
    F_{Q_1}(k \cos\theta_1, k \sin\theta_1) e^{i ( k \cos\theta_1 \delta_{x1} + k \sin\theta_1 \delta_{y1} )} =
    F_{Q_2}(k \cos\theta_2, k \sin\theta_2) e^{i ( k \cos\theta_2 \delta_{x2} + k \sin\theta_2 \delta_{y2} )}
\end{equation}
The above simplifies to:
\begin{equation}
    k (\cos\theta_1 \delta_{x1} + \sin\theta_1 \delta_{y1}) = k (\cos\theta_2 \delta_{x2} + \sin\theta_2 \delta_{y2})
\end{equation}
\begin{equation}
    \cos\theta_1 \delta_{x1} + \sin\theta_1 \delta_{y1} = \cos\theta_2 \delta_{x2} + \sin\theta_2 \delta_{y2}
\end{equation}

Note that the equation above provides one constraint on the unknown shifts. Given multiple particle images $\{Q_i\}_{i=1}^{N}$, we can generalize this approach:\\
For any two images $Q_i$ and $Q_j$:
\begin{equation}
    \cos\theta_i \delta_{xi} + \sin\theta_i \delta_{yi} = \cos\theta_j \delta_{xj} + \sin\theta_j \delta_{yj}
\end{equation}
For N images, there are $2N$ unknowns ($\delta_{xi}, \delta_{yi}$ for each image). wherein each unique image pair gives a new equation. Forming equations for $\frac{N(N-1)}{2}$ pairs provides a system of equations that can be solved.
Thus, solving this system recovers all translations $(\delta_{xi}, \delta_{yi})$.

\end{enumerate}
\end{document}

    