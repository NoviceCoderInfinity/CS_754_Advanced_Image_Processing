\documentclass{article}
\usepackage{helvet}


\usepackage{cite}
\usepackage{amsmath,amssymb,amsfonts}
\usepackage{algorithmic}
\usepackage{graphicx}
\usepackage{textcomp}
\usepackage{xcolor}
\usepackage{hyperref}
\usepackage{placeins}
\usepackage{graphicx}
\usepackage{subcaption}



\def\BibTeX{{\rm B\kern-.05em{\sc i\kern-.025em b}\kern-.08em
    T\kern-.1667em\lower.7ex\hbox{E}\kern-.125emX}}


\title{Assignment 2: CS 754, Spring 2024-25}
\author{
\IEEEauthorblockN{
    \begin{tabular}{cccc}
        \begin{minipage}[t]{0.23\textwidth}
            \centering
            Amitesh Shekhar\\
            IIT Bombay\\
            22b0014@iitb.ac.in
        \end{minipage} & 
        \begin{minipage}[t]{0.23\textwidth}
            \centering
            Anupam Rawat\\
            IIT Bombay\\
            22b3982@iitb.ac.in
        \end{minipage} & 
        \begin{minipage}[t]{0.23\textwidth}
            \centering
            Toshan Achintya Golla\\
            IIT Bombay\\
            22b2234@iitb.ac.in
        \end{minipage} \\
        \\ 
    \end{tabular}
}
}

\date{February 16, 2025}


\usepackage{amsmath}
\usepackage{amssymb}
\usepackage{hyperref}
\usepackage{ulem,graphicx}
\usepackage[margin=0.5in]{geometry}

\begin{document}
\maketitle

\textbf{Declaration:} The work submitted is our own, and
we have adhered to the principles of academic honesty while completing and submitting this work. We have not
referred to any unauthorized sources, and we have not used generative AI tools for the work submitted here.

\begin{enumerate}
\item A matrix $\boldsymbol{A} \in \mathbb{R}^{m \times n}$ where $m < n$ is said to satisfy the modified null space property (MNSP) relative to a set \textcolor{blue}{$S \subset [n] := \{1,2,...,n\}$} if for all $\boldsymbol{v} \in \textrm{nullspace}(\boldsymbol{A}) - \{\boldsymbol{0}\}$, we have \textcolor{blue}{$\|\boldsymbol{v}_S\|_1 < \|\boldsymbol{v}_{\bar{S}}\|_1$ where $\bar{S}$ stands for the complement of the set $S$}. The matrix $\boldsymbol{A}$ is said to satisfy MNSP of order $s$ if it satisfies the MNSP relative to any set $S \subset [n]$ where $|S| \leq s$. Now answer the following questions:
\begin{enumerate}
\item Consider a given matrix $\boldsymbol{A}$  and $\boldsymbol{v} \in \textrm{nullspace}(\boldsymbol{A}) - \{\boldsymbol{0}\}$. Suppose 
the condition $\|\boldsymbol{v}_S\|_1 \leq \|\boldsymbol{v}_{\bar{S}}\|_1$ is true for set $S$ that contains the indices of the $s$ largest absolute value entries of $\boldsymbol{v}$. Then is this condition also true for any other set $S$ such that $|S| \leq s$? Why (not)?
\item Show that the MNSP implies that $\|\boldsymbol{v}\|_1 < 2\sigma_{s,1}(\boldsymbol{v})$ for $\boldsymbol{v}  \in \textrm{nullspace}(\boldsymbol{A}) - \{\boldsymbol{0}\}$ where $\sigma_{s,1}(\boldsymbol{v}) := \textrm{inf}_{\|\boldsymbol{w}\|_0 \leq s} \|\boldsymbol{v} - \boldsymbol{w}\|_1$. 
\item Given a matrix $\boldsymbol{A} \in \mathbb{R}^{m \times n}$ of size $m \times n$, any $s$-sparse vector $\boldsymbol{x} \in \mathbb{R}^n$ is a unique solution of the P1 problem with the constraint $\boldsymbol{y} = \boldsymbol{Ax}$ if and only if $\boldsymbol{A}$ satisfies the MNSP of order $s$. 
\end{enumerate}
\textsf{[4+4+8=16 points]}
\\
            \makebox[0pt][l]{\hspace{-7pt}\textit{Soln:}} % Aligns "Answer:" to the left
\begin{enumerate}
    \item Yes, this condition does necessarily hold for any other subset \( S \) with \( |S| \leq s \). Let $\boldsymbol{v} \in \textrm{nullspace}(\boldsymbol{A}) - \{\boldsymbol{0}\}$ and define \( S^* \) as the set containing the indices of the \( s \) largest (in absolute value) entries of \( v \). By the construction of \( S^* \), for any other subset \( S \subset \{1,2,\dots,n\} \) with \( |S| \le s \), we have
\[
\|v_S\|_1 \leq \|v_{S^*}\|_1.
\]

We are given that
\[
\|v_{S^*}\|_1 \leq \|v_{\overline{S^*}}\|_1,
\]
where \( \overline{S^*} \) denotes the complement of \( S^* \) in \( \{1,2,\dots,n\} \).

Notice that for any such \( S \) with \( |S| \le s \), the complement \( \overline{S} \) satisfies
\[
\|v_{\overline{S}}\|_1 \geq \|v_{\overline{S^*}}\|_1,
\]
since \( S^* \) consists of the indices corresponding to the largest absolute values of \( v \), and hence \( \overline{S} \) must contain at least as much of the remaining (smaller) contributions as \( \overline{S^*} \) does.

Combining the above inequalities, we obtain:
\[
\|v_S\|_1 \leq \|v_{S^*}\|_1 \leq \|v_{\overline{S^*}}\|_1 \leq \|v_{\overline{S}}\|_1.
\]

Thus, the condition
\[
\|v_S\|_1 \leq \|v_{\overline{S}}\|_1
\]
holds for any subset \( S \) with \( |S| \le s \).
    \item We first recall that for a given vector \(\boldsymbol{v}\), its best \(s\)-term approximation error in the \(\ell_1\)-norm is defined as 
\[
\sigma_{s,1}(\boldsymbol{v}) := \inf_{\|\boldsymbol{w}\|_0 \le s} \|\boldsymbol{v} - \boldsymbol{w}\|_1.
\]
A natural choice is to take \(\boldsymbol{w}\) as the vector \(\boldsymbol{v}_S\) obtained by retaining the \(s\) largest (in absolute value) entries of \(\boldsymbol{v}\) and setting the rest to zero. If we denote by \(S\) the index set corresponding to these entries, then
\[
\sigma_{s,1}(\boldsymbol{v}) = \|\boldsymbol{v}_{\bar{S}}\|_1,
\]
where \(\bar{S}\) denotes the complement of \(S\).
 \\

Assume that \(\boldsymbol{A}\) satisfies the Modified Null Space Property (MNSP) of order \(s\); that is, for every set \(S' \subset [n]\) with \(|S'|\le s\) and every nonzero \(\boldsymbol{v} \in \textrm{nullspace}(\boldsymbol{A})\), we have
\[
\|\boldsymbol{v}_{S'}\|_1 < \|\boldsymbol{v}_{\bar{S'}}\|_1.
\]
Let \(\boldsymbol{v}\in \textrm{nullspace}(\boldsymbol{A})\setminus \{0\}\) and choose \(S\) to be the set of indices corresponding to the \(s\) largest (in absolute value) entries of \(\boldsymbol{v}\). Then, by the MNSP,
\[
\|\boldsymbol{v}_S\|_1 < \|\boldsymbol{v}_{\bar{S}}\|_1.
\]
Thus, we can write
\[
\|\boldsymbol{v}\|_1 = \|\boldsymbol{v}_S\|_1 + \|\boldsymbol{v}_{\bar{S}}\|_1 < \|\boldsymbol{v}_{\bar{S}}\|_1 + \|\boldsymbol{v}_{\bar{S}}\|_1 = 2\|\boldsymbol{v}_{\bar{S}}\|_1
\]
So,
\[
\|\boldsymbol{v}\|_1 < 2\|\boldsymbol{v}_{\bar{S}}\|_1
\]
Since \(\sigma_{s,1}(\boldsymbol{v}) = \|\boldsymbol{v}_{\bar{S}}\|_1\), it follows that
\[
\|\boldsymbol{v}\|_1 < 2\,\sigma_{s,1}(\boldsymbol{v}).
\]
Thus, the desired inequality holds.

\item 
Let $A\in\mathbb{R}^{m\times n}$ and fix an integer $s\ge 1$. Then, every $s$-sparse vector $x\in\mathbb{R}^n$ is the unique solution of
\[
\min_{z\in\mathbb{R}^n} \|z\|_1 \quad\text{subject to } Az=Ax\text{ and } y=Ax
\]
if and only if $A$ satisfies the MNSP of order $s$.

We prove the above statement by establishing both directions.\\
Suppose that $A$ satisfies the MNSP of order $s$. Let $x\in\mathbb{R}^n$ be any $s$-sparse vector with support
\[
S = \{i : x_i\neq 0\}, \quad |S|\le s.
\]
Assume that $z\in\mathbb{R}^n$ satisfies
\[
Az = Ax.
\]
Then we can write
\[
z = x + v \quad \text{with } v\in \mathrm{null}(A).
\]
If $v=0$, then $z=x$. If $v\neq 0$, note that
\[
z_S = x_S + v_S \quad \text{and} \quad z_{\bar{S}} = v_{\bar{S}}.
\]
By the triangle inequality we have
\[
\|z_S\|_1 \ge \|x_S\|_1 - \|v_S\|_1.
\]
Thus,
\[
\|z\|_1 = \|z_S\|_1 + \|z_{\bar{S}}\|_1 \ge \|x_S\|_1 - \|v_S\|_1 + \|v_{\bar{S}}\|_1.
\]
Since $A$ satisfies the MNSP relative to $S$, we know that
\[
\|v_S\|_1 < \|v_{\bar{S}}\|_1.
\]
It follows that
\[
\|z\|_1 > \|x_S\|_1 = \|x\|_1.
\]
Hence, any feasible $z\neq x$ has a strictly larger $\ell_1$ norm than $x$, so $x$ is the unique minimizer.
\\
Assume that any $s$-sparse vector $x\in\mathbb{R}^n$ is the unique solution of
\[
\min_{z\in\mathbb{R}^n}\|z\|_1 \quad \text{subject to} \quad Az = Ax.
\]
We will show that $A$ must satisfy the MNSP of order $s$. We argue by contradiction.

Suppose that $A$ does \emph{not} satisfy the MNSP of order $s$. Then there exists a set $S\subset [n]$ with $|S|\le s$ and a nonzero vector $v\in nullspace(A)$ such that
\[
\|v_S\|_1 \ge \|v_{\bar{S}}\|_1.
\]
Now, choose any $s$-sparse vector $x\in\mathbb{R}^n$ with support exactly $S$. Since $v\in nullspace(A)$, both x and z = x+v
satisfy $Az = Ax$. We now compare their $\ell_1$ norms.

Since $x$ is supported on $S$, we have
\[
\|x\|_1 = \|x_S\|_1.
\]
For the vector $z$, note that
\[
z_S = x_S+v_S\quad \text{and}\quad z_{\bar{S}} = v_{\bar{S}}.
\]
Thus, by the triangle inequality,
\[
\|z_S\|_1 \ge \|x_S\|_1 - \|v_S\|_1,
\]
so that
\[
\|z\|_1 = \|z_S\|_1 + \|z_{\bar{S}}\|_1 \ge \|x_S\|_1 - \|v_S\|_1 + \|v_{\bar{S}}\|_1
\]
because \[ \|z_{\bar{S}}\|_1 = \|v_{\bar{S}}\|_1 \]
Using the assumption $\|v_S\|_1 \ge \|v_{\bar{S}}\|_1$, it follows that
\[
\|z\|_1 \ge \|x_S\|_1 = \|x\|_1.
\]
Since $v\neq 0$, we have $z\neq x$. Hence, there exists a distinct feasible vector $z$ with
\[
\|z\|_1 \le \|x\|_1,
\]
which contradicts the uniqueness of $x$ as the minimizer. 

Thus, our assumption must be false, and $A$ must satisfy the MNSP of order $s$.
\end{enumerate}
\end{enumerate}
\end{document}