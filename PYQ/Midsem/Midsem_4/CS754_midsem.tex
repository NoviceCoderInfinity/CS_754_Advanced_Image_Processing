\title{Midsem: CS 754, Advanced Image Processing, 28th Feb}
\date{}
\documentclass[12pt]{article}

\usepackage{amsmath}
\usepackage{amssymb}
\usepackage{hyperref}
\usepackage[margin=0.2in]{geometry}
\usepackage{graphicx}
\usepackage{epstopdf}

\newcommand*\Laplace{\mathop{}\!\mathbin\bigtriangleup}
\begin{document}
\vspace{-1in}
\maketitle
\vspace{-0.3in}
\textbf{Instructions:} There are 120 minutes for this exam. This exam is worth 10\% of your final grade. Attempt all questions. Avoid writing lengthy answers. Each question carries 10 points. 

\begin{enumerate}
\item Let $\boldsymbol{\theta^{\star}}$ be the result of the following minimization problem: $\textrm{min} \|\boldsymbol{\theta}\|_1$ such that $\|\boldsymbol{y}-\boldsymbol{\Phi \Psi \theta}\|_2 \leq \varepsilon$, where $\boldsymbol{y}$ is an $m$-element measurement vector, $\boldsymbol{\Phi}$ is a $m \times n$ measurement matrix ($m < n$), $\boldsymbol{\Psi}$ is a $n \times n$ orthonormal basis in which $n$-element signal $\boldsymbol{x}$ has a sparse representation of the form $\boldsymbol{x} = \boldsymbol{\Psi \theta}$. Notice that $\boldsymbol{y} = \boldsymbol{\Phi x} + \boldsymbol{\eta}$ and $\varepsilon$ is an upper bound on the magnitude of the noise vector $\boldsymbol{\eta}$.

Theorem 3 we studied in class states the following: If $\boldsymbol{\Phi}$ obeys the restricted isometry property with isometry constant $\delta_{2s} < \sqrt{2}-1$, then we have $\|\boldsymbol{\theta} - \boldsymbol{\theta^{\star}}\|_2 \leq C_1 s^{-1/2}\|\boldsymbol{\theta}-\boldsymbol{\theta_s}\|_1 + C_2 \varepsilon$ where $C_1$ and $C_2$ are functions of only $\delta_{2s}$ and where $\forall i \in \mathcal{S}, \boldsymbol{\theta_s}_i = \theta_i; \forall i \notin \mathcal{S}, \boldsymbol{\theta_s}_i = 0$. Here $\mathcal{S}$ is a set containing the $s$ largest magnitude elements of $\boldsymbol{\theta}$. 

A curious (and thus, very good) student asks the following questions: `(1) It appears that the error bound is reduced as $s$ increases, which goes against the very premise of compressed sensing. How do we address this apparent discrepancy? (2) It also appears that the error bound is independent of $m$. How do you address this?' Your job is to answer the student's questions.

\item Write down the objective function (with the meaning of each term clearly stated) for compressed sensing based, coupled tomographic reconstruction of three structurally similar 2D image slices $\boldsymbol{x_1}, \boldsymbol{x_2}, \boldsymbol{x_3}$ (of equal size) from their respective tomographic measurements $\boldsymbol{y_1} = \boldsymbol{R_1 x_1}, \boldsymbol{y_2} = \boldsymbol{R_2 x_2}, \boldsymbol{y_3} = \boldsymbol{R_3 x_3}$ where $\boldsymbol{R_1}, \boldsymbol{R_2}, \boldsymbol{R_3}$ are the respective Radon matrices. State the advantages of the coupled reconstruction over independent reconstruction. What would happen to the coupled reconstruction if $\boldsymbol{R_1} = \boldsymbol{R_2} = \boldsymbol{R_3}$? What would happen to the coupled reconstruction if the three slices are not structurally similar?

\item While random matrices from certain distributions are useful, at least theoretically, as sensing matrices for compressive reconstructions, we have seen in class that it is useful to \emph{design} matrices that are optimal in terms of certain properties - such as the mutual coherence $\mu$ or the restricted isometry property (RIP). What are the relative advantages and disadvantages of $\mu$ versus RIP in this context?

\item State the Fourier slice theorem for 3D images and 2D tomographic projections. 

\item What is the intuitive reason that $\ell_1$ optimization is preferred over $\ell_2$ optimization in compressive reconstruction? You may draw a diagram if required. 

\item Explain the term `common lines' in the context of tomography of 3D objects under unknown projection angles. What is their application (state very briefly in one or two sentences)?

\item The order $s$ restricted isometry constant $\delta_s$ of a sensing matrix $\boldsymbol{\Phi}$ is the smallest constant such that for any $s$-sparse vector $\boldsymbol{x}$, we have $(1-\delta_s) \|\boldsymbol{x}\|^2 \leq \|\boldsymbol{\Phi x}\|^2 \leq (1+\delta_s) \|\boldsymbol{x}\|^2$. If $\delta_s < 1$, we say that $\boldsymbol{\Phi}$ obeys the restricted isometry property (RIP) of order $s$. If a matrix obeys RIP, we know that no $s$-sparse vector can lie in its null-space. Is the converse also true? That is, if no $s$-sparse vector lies in the nullspace of $\boldsymbol{\Phi}$, does that imply that $\boldsymbol{\Phi}$ obeys RIP? Justify.
\end{enumerate}


\end{document}